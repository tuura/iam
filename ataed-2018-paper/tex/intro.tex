Deciding whether two given events in a trace are \emph{concurrent}, i.e. have
no causal or data dependencies between them, is a major problem in the process
discovery field~\cite{2011_aalst_book}. Various methods for concurrency
extraction, often referred to as \emph{concurrency oracles}, have been
introduced, including the classic $\alpha$-algorithm~\cite{van2004workflow}, as
well as a few less widely known methods, e.g. see~\cite{cook1998event},
\cite{mokhov2016mining} and the review paper~\cite{augusto2017automated}. A
good example of treating a concurrency oracle as a self-contained problem can
be found in~\cite{dumas2015process}.

In this paper we present an approach for deriving concurrency oracles for
events that correspond to processor instructions and blocks of instructions. The
input to the proposed approach is the \emph{microarchitectural semantics} of
instructions, which gives a precise description of how an instruction execution
changes the state of the processor.

A popular method to describe microarchitectural semantics is to use a dedicated
domain-specific language embedded in a high-level general-purpose host language,
such as Haskell or Coq. Two pioneering works in this domain
are~\cite{fox2010trustworthy}, where the Arm v7 instruction set architecture
is formalised in HOL4, and~\cite{kennedy2013coq}, where x86 architecture is
formalised in Coq.

The authors of this paper have also used an embedded domain-specific language
to describe the semantics of a space-grade
microarchitecture~\cite{mokhov2018formal}. In particular, it was demonstrated
that the same semantics can be reused in different contexts: to simulate the
processor and to perform formal verification of programs executed by the
processor. In this paper we take this work further, by demonstrating that the
very same semantics can be reused for deriving concurrency oracles that given
two instructions, or blocks of instructions, can determine whether they are
concurrent and, if not, report the data dependency conflicts.

We start by studying several common examples of processor instructions, noticing
that different instructions require different features from the language used to
describe them; see~\S\ref{sec:instructions}. We proceed by introducing the
language for specifying semantics in more detail and then describe the semantics
of a small instruction set (\S\ref{sec:metalanguage}). The approach to deriving
concurrency oracles is presented in~\S\ref{sec:oracles}, followed by a
discussion.
