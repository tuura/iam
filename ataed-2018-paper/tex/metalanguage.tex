
In the previous section we have described the semantics of several instructions of
a generic computer architecture in terms of a polymorphic computational metalanguage.
This section presents the formal definition of the metalanguage and provides a more
formal description of the instruction and program semantics.

\textbf{A remark on formal definitions}
Before we start, let us make a remark on what we consider a formal definition.
We do not aim to formalise our tools in any kind of foundational mathematical system,
such as ZF~\footnote{Zermelo-Fraenkel set theory} or homotopy type theory. We are
presenting an elegant way of solving a well-known problem and we use the Haskell
programming language to implement the solution. Therefore, we consider a concept
to be~\emph{formally} defined if it is expressed as a Haskell data type. This may
sound hand-wavy, but since Haskell has a static type system (a variant of System F)
and operational semantics, we can be formal enough.

\textbf{Definition (polymorphic computational metalanguage):\label{def:metalanguage}}
The metalanguage is a value of the following Haskell type:

\begin{minted}{haskell}
type Semantics c a =
    forall f. c f => (Key -> f Value)
                  -> (Key -> f Value -> f ())
                  -> Maybe (f a)
\end{minted}

\noindent A~\hs{Semantics} is essentially a rank-2 polymorphic\footnote{A rank-2 polymorphic
function is one taking as a parameter another function, which is in turn (rank-1)
polymorphic. This feature requires~\hs{RankNTypes} GHC extension to be enabled.}
effectful computation depending on two functions,
which we will usually referred to as~\hs{read} and~\hs{write}.

Let us now give some intuition for the components of the metalanguage.
The~\hs{Semantics c a}
type may be thought as a mutable dictionary. The~\hs{read} function has
type~\hs{Key -> f Value} --- it takes
a key and gives back an effectful value looked up in the dictionary. The~\hs{write}
function takes a key and an effectful value and alters the value of the key.
The semantics is allowed to fail to produce a result, hence the
the return type~hs{f a} is wrapped in the~\hs{Maybe} type constructor.~\hs{Maybe}
\footnote{This type is defined in the Haskell base library as~\hs{data Maybe a = Just a | Nothing}}
is an idiomatic Haskell encoding of potentially faulty computations. We give the
semantics a possibility of failure in order to avoid partial functions. The semantics
may become partial if we fix the constraint type variable~\hs{c} to~\hs{Applicative},
thus losing possibility to encode the monadic components of the instruction set.
~\hs{Maybe} gives us a possibility to mock the encoding of monadic instructions
and make the semantics total.

\textbf{Definition (instruction set):} An~\emph{instruction set} is an algebraic data type
comprising as much data constructors as there are instructions.
If an instruction has an argument, it is defined as an argument of the corresponding data constructor.

Consider an example definition of an instruction set consisting
of instructions described in the previous sections and the related auxiliary types:

\begin{minted}{haskell}
data Instruction = Load     Register MemoryAddress
                 | LoadMI   Register MemoryAddress
                 | Add      Register MemoryAddress
                 | Jump     Value

data Register = R0 | R1 | R2 | R3

type MemoryAddress = Value
\end{minted}

\textbf{Definition (instruction set semantics):}

The semantics of an instruction set is a Haskell function mapping
data constructors of the instruction sets to the terms of the polymorphic
computational metalanguage.

The definition of an instruction set semantics is the point where the metalanguage
has to be made monomorphic, i.e. the context constraint has to be instantiated with
a concrete one. Now we will consequently present unrestricted, functorial,
applicative and monadic semantics for the defined instruction set:

We start from the~\hs{load} instruction which may be executed in any
context\footnote{The~\hs{Unrestricted} constraint is not exactly idiomatic Haskell and
requires some tricks to be defined.}:

\begin{minted}{haskell}
semanticsU :: Instruction -> Semantics Unrestricted ()
semanticsU (Load reg addr)   = load reg addr
semanticsU _                 = const (const Nothing)
\end{minted}

\noindent Note how the Haskell wildcard pattern to rule out instructions that require a more
restrictive context. The~\hs{const (const Nothing)} expression is equivalent
to~\hs{\_ -> \_ -> Nothing} and constructs a stub for these more restricted
semantics.

The instantiation of~\hs{c} with a~\hs{Functor} permits to implement the semantics
of~\hs{jump}:

\begin{minted}{haskell}
semanticsF :: Instruction -> Semantics Functor ()
semanticsF i@(Load _ _)      = semanticU i
semanticsF (Jump simm)       = jump simm
semanticsF _                 = const (const Nothing)
\end{minted}

\noindent Here we reuse the unrestricted semantics for the~\hs{load} instruction.

The rest of the semantic definitions are straitforward:

\begin{minted}{haskell}
semanticsA :: Instruction -> Semantics Applicative ()
semanticsA i@(Load _ _)      = semanticU i
semanticsA i@(Jump simm)     = semanticA i
semanticsA (Add reg addr)    = add reg addr
semanticsA _                 = const (const Nothing)

semanticsM :: Instruction -> Semantics Monad ()
semanticsM (LoadMI reg addr) = loadMI reg addr
semanticsM i                 = semanticsA i
\end{minted}

We can now define the semantics of a block of instructions by reducing
the lost of instructions:

\begin{minted}{haskell}
blockSemanticsA :: [Instruction] -> Semantics Applicative ()
blockSemanticsA xs = \r w ->
    foldr (\x acc -> ((*>)) <$> acc <*> semanticsA x r w) nop xs
    where nop = Just $ pure ()
\end{minted}

\noindent The semantics of an empty block is~\hs{nop}. The semantics of a non-empty
block is the semantics of its head chained with the semantics of the tail. We
need to lift the applicative chaining operation since the~\hs{Maybe} type constructor
also is an instance of~\hs{Applicative} and the behaviour of~\hs{*>} returns
the contents of the last~\hs{Just}, which is would be wrong.

Now, with the instruction semantics defined in terms of the polymorphic
computational metalanguage, we may proceed to evaluating the metalanguage
in concrete contexts to get a practical interpretation of the instruction set.

The next section presents the interpretation of unrestricted, functorial and
applicative instructions yielding concurrency oracles for programs.

The other interpretations of the metalanguage are also possible. In the technical
report~\cite{mokhov2018formal} we present a formal model of a custom computer
architecture developed by RUAG Space Austria GmbH for space missions.
This work uses a more restricted, monadic metalanguage and makes an accent on
symbolic execution and automated theorem proving: the framework allows to verify
functional properties of programs and automatically check if two programs are
semantically equivalent. Potentially, the metalanguage presented in this paper
allows these interpretations.