% The instruction set of IAM comprises 8 instructions. Consider the following
% instruction mnemonics and informal descriptions of their behaviour:

% \begin{longtable}{l|p{9cm}}
% \texttt{load r memaddr}     & Load a value from a memory location to a register.\\
% \texttt{loadmi r memaddr}   & Load a value from the memory location to the register using
% an indirect memory access mode\footnote{Loading the value from a memory location and using it as
% a memory address argument for the~\texttt{load} instruction.}.\\
% \texttt{set      r byte   } & Load an 8-bit immediate argument to a register.\\
% \texttt{store    r memaddr} & Store a value from a register to a memory location.\\
% \texttt{add      r memaddr} & Add a value placed in a memory location to a value contained in a register.\\
% \texttt{jump     byte     } & Perform an unconditional jump modifying the machine
% instruction counter by a given offset.\\
% \texttt{jumpz    byte     } & Performs a conditional jump if the zero flag is set.\\
% \texttt{halt              } & Stop the execution setting the halt flag.
% \end{longtable}

Figure~\ref{fig:IAMtypes} presents and encoding of the IAM microarchitecture
as a collection of Haskell types.

\begin{figure}[t]
\caption{Haskell data types encoding IAM microarchitecture.\label{fig:IAMtypes}}
\begin{minted}{haskell}
data MachineState = MachineState
    { registers           :: RegisterBank
    , instructionCounter  :: InstructionAddress
    , instructionRegister :: Instruction
    , flags               :: Flags
    , memory              :: Memory
    , program             :: Program
    , clock               :: Clock
    }

type Value = Byte

data Register = R0 | R1 | R2 | R3
type RegisterBank = Map Register Value

type MemoryAddress = Value
type Memory = Map MemoryAddress Value

data Flag = Zero
          | Halted
type Flags = Map Flag Bool

type Clock = Value
\end{minted}
\end{figure}
