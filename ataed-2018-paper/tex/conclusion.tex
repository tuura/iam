% \begin{wrapfigure}{r}{0.4\textwidth}
%   % \begin{center}
%     \vspace{-25mm}
%     \includegraphics[scale=0.4]{img/.pdf}
%   % \end{center}
%   \caption{Approximation of static dependencies of an array summation program.\label{fig-sum}}
% \end{wrapfigure}

% The metalanguage closely resembles state
% transformers.
The paper presented a metalanguage for describing the semantics of instruction
set architectures. Multiple interpretations of the metalanguage terms allow us
to evaluate the semantics in different contexts without its modification. As the
primary application, we present an approach to deriving concurrency oracles of
the instructions with only static data dependencies.

To handle instructions whose dependencies are dynamic, it is possible to use
conservative approximation of dependencies. For example, Fig.~\ref{fig-sum}
shows the dependency graph for a program implementing the Euclidean algorithm
for computing the greatest common divisor of two numbers that contains
a conditional branch instruction \hs{JumpZero}, which modifies the instruction
counter \hs{IC} only if the previous instruction set the \hs{Zero} flag. In this
example, we conservatively assume that \hs{JumpZero} always depends on \hs{IC}.
Our future work includes the application of the presented methodology to
extracting concurrency from real-world processor specifications, as described
in~\cite{mokhov2018formal}.

\vspace{-10mm}
\begin{figure}
\centerline{\includegraphics[scale=0.4]{img/gcd.pdf}}
\vspace{-7mm}
\caption{Approximation of static dependencies of the Euclidean algorithm.\label{fig-sum}}
\vspace{-10mm}
\end{figure}
% \clearpage
